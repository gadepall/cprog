\begin{enumerate}[label=\thesection.\arabic*, ref=\thesection.\theenumi]
\item Do the following addition through a C program
	$$17+23$$
	\\
	\solution
	\lstinputlisting{codes/integer/add.c}
\item Do the following subtraction through a C program
$$7-9$$
	\\
	\solution
	\lstinputlisting{codes/integer/sub.c}
\item Mulitply the following through a C program
	$$4\times \brak{-8}$$
	\\
	\solution
	\lstinputlisting{codes/integer/mul.c}
\item Perform the following division
	$$\brak{-100}\div 5$$
	\\
	\solution
	\lstinputlisting{codes/integer/div.c}
\end{enumerate}
Compute the following
\begin{enumerate}[label=\thesection.\arabic*, ref=\thesection.\theenumi,resume*]
	\begin{multicols}{4}
\item $\brak{-75}+18$
\item $19+\brak{-25}$
\item $27+\brak{-27}$
\item $\brak{-20}+0$
\item $\brak{-35}+\brak{-10}$
\item $\brak{-10}+3$
\item $17-(-21)$
\item $8\times(-2)$
\item $3\times(-7)$
\item $10\times(-1)$
\item $6\times(-19)$
\item $12\times(-32)$
\item $7\times(-22)$
\item $15\times(-16)$
\item $21\times(-32)$
\item $(-42)\times 12$
\item $(-55)\times 15$
\item $(-5)\times \brak{-6}$
\item $\brak{-6}\times(-7)$
\item $3\times(-1)$
\item $\brak{-1}\times 225$
\item $\brak{-21}\times(-30)$
\item $\brak{-316}\times(-1)$
\item	$\brak{-81}\div 9$
\item	$\brak{-75}\div 5$
\item	$\brak{-32}\div 2$
\item	$125\div \brak{-25}$
\item	$80\div \brak{-5}$
\item	$64\div \brak{-16}$
\item	$\brak{-30}\div 10$
\item	$50\div \brak{-5}$
\item	$\brak{-36}\div \brak{-9}$
\item	$\brak{-49}\div \brak{-49}$
\end{multicols}
\end{enumerate}
\begin{enumerate}[label=\thesection.\arabic*, ref=\thesection.\theenumi,resume*]
	\begin{multicols}{2}
\item	$13\div \sbrak{\brak{-2}+1}$
\item	$\brak{-31}\div \sbrak{\brak{-30}+\brak{-1}}$
\item	$\sbrak{\brak{-36}\div 12}\div \brak{3}$
\item	$\sbrak{\brak{-6}+5}\div \sbrak{\brak{-2}+1}$
	\end{multicols}
\end{enumerate}
Fill in the blanks
\begin{enumerate}[label=\thesection.\arabic*, ref=\thesection.\theenumi,resume*]
	\begin{multicols}{2}
		\item	$20 \div \rule{1cm}{0.1pt}=-2$
		\item	$\rule{1cm}{0.1pt}\div 4=-3$
	\end{multicols}
\end{enumerate}
Find the values of the following expressions for $x = 2$. 
\begin{enumerate}[label=\thesection.\arabic*, ref=\thesection.\theenumi,resume*]
\item $x + 4$
\item  $4x – 3$ 
\item $ x – 2$
\item $ 3x – 5 $
\item $ 9 – 5x $
\item $ x + 7 + 4 (x – 5)$
\item $ 3 (x + 2) + 5x – 7 $
\item $ 6x + 5 (x – 2) $
\item $ 4(2x – 1) + 3x + 11$
\end{enumerate}
If $x = – 2$, find the value of
\begin{enumerate}[label=\thesection.\arabic*, ref=\thesection.\theenumi,resume*]
\item $5x – 2$
\item $4p + 7$
\end{enumerate}
Find the value of the following expressions for $a = 3, b = 2$. 
\begin{enumerate}[label=\thesection.\arabic*, ref=\thesection.\theenumi,resume*]
\item $a + b$				
\item $ 7a – 4b $
\end{enumerate}
Find the value of the following expressions, when $x = –1$ 
\begin{enumerate}[label=\thesection.\arabic*, ref=\thesection.\theenumi,resume*]
\item $ 2x – 7$
\item $ – x + 2$ 
\end{enumerate}
When $a = 0, b = – 1$, find the value of the given expressions
\begin{enumerate}[label=\thesection.\arabic*, ref=\thesection.\theenumi,resume*]
\item $2a + 2b$
\end{enumerate}
Simplify these expressions and find their values if $x = 3, a = – 1, b = – 2$. 
\begin{enumerate}[label=\thesection.\arabic*, ref=\thesection.\theenumi,resume*]
\item $ 2x +4 $
\item $6 - 4x$
\item $6 – 5a $
\item $6 – 8b $
\item $3a – 2b-9 $
\end{enumerate}
\begin{enumerate}[label=\thesection.\arabic*, ref=\thesection.\theenumi,resume*]
	\item In a test (+5) marks are given for every correct answer and (-2) marks for every incorrect answer.  
		\begin{enumerate}
			\item Radhika answered all the questions and scored 30 marks though she got 10 correct answers.
			\item Jay also answered all the questions and scored (-12) marks though he got 4 correct answers.  How many incorrect answers had they  attempted?
		\end{enumerate}
	\item A shopkeeper earns a profit of \rupee 1 by selling one pen and incurs a loss of 40 paise per pencil while selling pencils of her old stock.  
		\begin{enumerate}
			\item In a particular month she incurs a loss of \rupee 5.  In this period she sold 45 pens.  How many pencils did she sell in this period?
			\item In the next month she earns neither profit nor loss.  If she sold 70 pens, how many pencils did she sell?
		\end{enumerate}
			\item The temperature at 12 noon was 10\degree C above zero. If it decreases at the rate of 2\degree C per hour unitl midnight, at what time would the temperature be 8\degree C below zero? 
	\item In a class test (+3) marks are given for every correct answer and (-2) marks for every incorrect answer and no marks for not attempting any question.   
		\begin{enumerate}
			\item Radhika scored 20 marks.  If she got 12 correct answers, how many questions has she attempted incorrectly?
			\item Mohini scores -5 marks in this test, though she has got 7 correct answers.   How many questions has she attempted incorrectly?
		\end{enumerate}
	\item An elevator descends a mine shaft at the rate of $6m/min$.  If the descent starts from $10m$ above the ground, how long will it take to reach $-350m$.
	\item What is the measure of the complement of each of the following angles? 
		\begin{enumerate}
			\begin{multicols}{4}
	\item 45\degree
\item 65\degree 
\item 41\degree 
\item 54\degree 
			\end{multicols}
\end{enumerate}
\item What will be the measure of the supplement of each one of the following angles? 
		\begin{enumerate}
			\begin{multicols}{4}
\item 100\degree
\item  90\degree 
\item  55\degree 
\item  125\degree
			\end{multicols}
\end{enumerate}
\item An exterior angle of a triangle is of measure $70\degree$ and one of its interior opposite angles is of measure 25\degree. Find the measure of the other interior opposite angle.
\item The two interior opposite angles of an exterior angle of a triangle are $60\degree$ and 80\degree. Find the measure of the exterior angle.
\item 	Two angles of a triangle are $30\degree$ and 80\degree. Find the third angle. 
\item  One of the angles of a triangle is $80\degree$ and the other two angles are equal. Find the measure of each of the equal angles.
\item  The three angles of a triangle are in the ratio 1:2:1. Find all the angles of the triangle. Classify the triangle in two different ways.
\item One of the sides and the corresponding height of a parallelogram are 4 cm and 3 cm respectively. Find the area of the parallelogram.
\item Find the height $x$ if the area of the parallelogram is 24 $cm^2$ and the base is 4 cm.
\item Find BC, if the area of the triangle ABC is 36 $cm^2$ and the height AD is 3 cm.
		\end{enumerate}

\begin{enumerate}[label=\thesubsection.\arabic*, ref=\thesubsection.\theenumi]
\item Do the following addition through a C program
	$$17+23$$
	\\
	\solution
	\lstinputlisting{codes/add.c}
\item Do the following subtraction through a C program
$$7-9$$
	\\
	\solution
	\lstinputlisting{codes/sub.c}
\end{enumerate}
Compute the following
\begin{enumerate}[label=\thesubsection.\arabic*, ref=\thesubsection.\theenumi,resume*]
\item $\brak{-75}+18$
\item $19+\brak{-25}$
\item $27+\brak{-27}$
\item $\brak{-20}+0$
\item $\brak{-35}+\brak{-10}$
\item $\brak{-10}+3$
\item $17-(-21)$
\end{enumerate}
In a quiz, team A scored $a_1 = -40, a_2=10, a_3=0$ and team B scored $b_1=10, b_2=0, b_3=-40$ in three successive rounds.
\begin{enumerate}[label=\thesubsection.\arabic*, ref=\thesubsection.\theenumi,resume*]
\item  If the total scores are 
	\begin{align}
		a &= a_1+a_2+a_3
		\\
		b &= b_1+b_2+b_3
	\end{align}
	which team scored more? 
	\\
	\solution 
	\lstinputlisting{codes/ifelse.c}
\item Write a function to compare the final scores.  Check for the cases when $a = -40, b = -40; a = 30, b = 20; a = -20, b = -10$.
	\\
	\solution 
	\lstinputlisting{codes/func.c}
\item Use arrays and a for loop to evaluate 
	\begin{align}
		a &= \sum_{i=0}^{2}a_i
		\\
		b &= \sum_{i=0}^{2}b_i
	\end{align}
	\\
	\solution 
	\lstinputlisting{codes/loop.c}
\item Revise the above code using only functions.
	\\
	\solution 
	\lstinputlisting{codes/loopfunc.c}
\item Use files for the input data.
	\\
	\solution 
	\lstinputlisting{codes/files.c}
\item Revise the files program using pointer arrays
	\\
	\solution 
	\lstinputlisting{codes/pointer.c}
\end{enumerate}

\begin{enumerate}[label=\thesubsection.\arabic*.,ref=\thesubsection.\theenumi]
\item Construct a $\triangle ABC$ given $a, \angle B$ and $K = b+c$.
		\label{prob:9/11/2/1}
	\\
	\solution 
	Using the cosine formula in  $\triangle ABC$,
\begin{align}
	{b}^2&= {a}^2 + {c}^2 - 2ac\cos{B}
\\
\implies	(K-c)^2 &= {a}^2 + c^2- 2  a  c\cos{B}
\\
\implies
	c &=
	\frac{K^2-a^2}{2\brak{K- a  \cos{B}}}
		\label{eq:9/11/2/1}
\end{align}
The coordinates of $\triangle ABC$ can then be expressed as
\begin{align}
		\label{eq:9/11/2/1-final}
	\vec{A}=c\myvec{\cos B \\ \sin B},
	\vec{B} = \vec{0},
	\vec{C} =\myvec{a \\ 0}.
\end{align}
\item Construct a $\triangle ABC$ given $\angle B, \angle C$ and $K = a+b+c$.
	\\
	\solution
	\begin{align}
a+b+c &= K \\
b\cos C + c \cos B -a &=0 \\
b\sin C - c \sin B &=0
\end{align}
resulting in the matrix equation
\begin{align}
		\label{eq:9/11/2/4}
	\myvec{1 & 1 & 1 \\ -1 & \cos C & \cos B  \\ 0 &\sin C & -\sin B } \myvec{a \\ b \\ c} = K \myvec{1 \\ 0 \\ 0}
\end{align}
which can be solved to obtain all the sides.  $\triangle ABC$ can then be plotted using
\begin{align}
\vec{A} = \myvec{a \\ b},\,
\vec{B} = \vec{0},\, 
\vec{C} = \myvec{a \\ 0}
		\label{eq:9/11/2/4-final}
\end{align}
\end{enumerate}

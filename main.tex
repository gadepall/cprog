\documentclass[journal]{IEEEtran}
\usepackage[a5paper, margin=10mm]{geometry}
%\usepackage{lmodern} % Ensure lmodern is loaded for pdflatex
\usepackage{tfrupee} % Include tfrupee package


\setlength{\headheight}{1cm} % Set the height of the header box
\setlength{\headsep}{0mm}     % Set the distance between the header box and the top of the text


%\usepackage[a5paper, top=10mm, bottom=10mm, left=10mm, right=10mm]{geometry}

%
\usepackage{gvv-book}
\usepackage{gvv}
%\setlength{\intextsep}{10pt} % Space between text and floats

\makeindex

\begin{document}
\bibliographystyle{IEEEtran}
\onecolumn


\title{
	%\begin{flushleft}
	\begin{center}
	%MATRICES \\ In Geometry
	C Programming in Middle School
%	Progressions
	\\
\rule{0.4\columnwidth}{0.4pt}
%\end{flushleft}
\end{center}
}
\author{
\vspace{11cm}
	%\begin{flushleft}
	\begin{center}
\includegraphics[width=0.2\columnwidth]{figs/logo.jpg}
\\
		{\huge	G. V. V. Sharma}\\Associate Professor,\\Department of Electrical Engineering, \\ IIT Hyderabad
	\end{center}
	%\end{flushleft}
%\IEEEpubid{\makebox[\columnwidth]{978-1-7281-5966-1/20/\$31.00 ©2020 IEEE \hfill} \hspace{\columnsep}\makebox[\columnwidth]{ }}
}
\maketitle

\newpage
\section*{About this Book}

This book introduces C programming for middle school children based on the
 NCERT mathematics textbook of Class 7.  

There is no copyright, so readers are free to print and share.  

This book is dedicated to my Hindi teacher in middle school, Shri Mandavi.
\begin{flushright}
\today
\end{flushright}
Github: https://github.com/gadepall/cprog
		\\
License: https://creativecommons.org/licenses/by-sa/3.0/
\\
and
\\
https://www.gnu.org/licenses/fdl-1.3.en.html

\newpage


\tableofcontents

\newpage
%\twocolumn
\onecolumn


%\renewcommand{\theequation}{\theenumi}
\numberwithin{equation}{enumi}
%\numberwithin{figure}{enumi}
%\numberwithin{figure}{section}
%\numberwithin{figure}{subsection}
\renewcommand{\thefigure}{\theenumi}
\renewcommand{\thetable}{\theenumi}

\section{Integers}
\subsection{Formulae}
\begin{enumerate}[label=\thesubsection.\arabic*, ref=\thesubsection.\theenumi]
\item Do the following addition through a C program
	$$17+23$$
	\\
	\solution
	\lstinputlisting{codes/add.c}
\item Do the following subtraction through a C program
$$7-9$$
	\\
	\solution
	\lstinputlisting{codes/sub.c}
\item Mulitply the following through a C program
	$$4\times \brak{-8}$$
	\\
	\solution
	\lstinputlisting{codes/mul.c}
\item Perform the following division
	$$\brak{-100}\div 5$$
	\\
	\solution
	\lstinputlisting{codes/div.c}
\end{enumerate}
Compute the following
\begin{enumerate}[label=\thesubsection.\arabic*, ref=\thesubsection.\theenumi,resume*]
	\begin{multicols}{4}
\item $\brak{-75}+18$
\item $19+\brak{-25}$
\item $27+\brak{-27}$
\item $\brak{-20}+0$
\item $\brak{-35}+\brak{-10}$
\item $\brak{-10}+3$
\item $17-(-21)$
\item $8\times(-2)$
\item $3\times(-7)$
\item $10\times(-1)$
\item $6\times(-19)$
\item $12\times(-32)$
\item $7\times(-22)$
\item $15\times(-16)$
\item $21\times(-32)$
\item $(-42)\times 12$
\item $(-55)\times 15$
\item $(-5)\times \brak{-6}$
\item $\brak{-6}\times(-7)$
\item $3\times(-1)$
\item $\brak{-1}\times 225$
\item $\brak{-21}\times(-30)$
\item $\brak{-316}\times(-1)$
	$\brak{-81}\div 9$
	$\brak{-75}\div 5$
	$\brak{-32}\div 2$
	$125\div \brak{-25}$
	$80\div \brak{-5}$
	$64\div \brak{-16}$
	\end{multicols}
\end{enumerate}

\subsection{NCERT}
\begin{enumerate}[label=\thesubsection.\arabic*, ref=\thesubsection.\theenumi]
\item Do the following addition through a C program
	$$17+23$$
	\\
	\solution
	\lstinputlisting{codes/add.c}
\item Do the following subtraction through a C program
$$7-9$$
	\\
	\solution
	\lstinputlisting{codes/sub.c}
\item Mulitply the following through a C program
	$$4\times \brak{-8}$$
	\\
	\solution
	\lstinputlisting{codes/mul.c}
\item Perform the following division
	$$\brak{-100}\div 5$$
	\\
	\solution
	\lstinputlisting{codes/div.c}
\end{enumerate}
Compute the following
\begin{enumerate}[label=\thesubsection.\arabic*, ref=\thesubsection.\theenumi,resume*]
	\begin{multicols}{4}
\item $\brak{-75}+18$
\item $19+\brak{-25}$
\item $27+\brak{-27}$
\item $\brak{-20}+0$
\item $\brak{-35}+\brak{-10}$
\item $\brak{-10}+3$
\item $17-(-21)$
\item $8\times(-2)$
\item $3\times(-7)$
\item $10\times(-1)$
\item $6\times(-19)$
\item $12\times(-32)$
\item $7\times(-22)$
\item $15\times(-16)$
\item $21\times(-32)$
\item $(-42)\times 12$
\item $(-55)\times 15$
\item $(-5)\times \brak{-6}$
\item $\brak{-6}\times(-7)$
\item $3\times(-1)$
\item $\brak{-1}\times 225$
\item $\brak{-21}\times(-30)$
\item $\brak{-316}\times(-1)$
	$\brak{-81}\div 9$
	$\brak{-75}\div 5$
	$\brak{-32}\div 2$
	$125\div \brak{-25}$
	$80\div \brak{-5}$
	$64\div \brak{-16}$
	\end{multicols}
\end{enumerate}

\section{Decimal Numbers}
\subsection{Formulae}
Find
\begin{enumerate}[label=\thesection.\arabic*,ref=\thesection.\theenumi,itemsep=1ex]
	\begin{multicols}{4}
%
	\item $\frac{2}{7}\times 3$
	\item $\frac{9}{7}\times 6$
	\item $\frac{1}{8}\times 3$
	\item $\frac{13}{11}\times 6$
	\item $\frac{2}{5}\times 2$
	\item $3\times 5\frac{1}{5}$
	\item $5\times 6\frac{3}{4}$
	\item $7\times 2\frac{1}{4}$
	\item $4\times 6\frac{1}{3}$
	\item $6\times 3\frac{1}{4}$
	\item $8\times 3\frac{2}{5}$
	\item $\frac{1}{2}\times \frac{1}{7}$
	\item $\frac{1}{5}\times \frac{1}{7}$
	\item $\frac{1}{3}\times \frac{4}{5}$
	\item $\frac{2}{3}\times \frac{1}{5}$
	\item $\frac{8}{3}\times \frac{4}{7}$
	\item $\frac{3}{4}\times \frac{2}{3}$
	\item $\frac{2}{3}\times 2\frac{2}{3}$
	\item $\frac{2}{7}\times \frac{7}{9}$
	\item $\frac{3}{8}\times \frac{6}{4}$
	\item $\frac{9}{5}\times \frac{3}{5}$
	\item $\frac{1}{3}\times \frac{15}{8}$
	\item $\frac{11}{2}\times \frac{3}{10}$
	\item $\frac{4}{5}\times \frac{12}{7}$
	\item $\frac{2}{5}\times 5\frac{1}{4}$
	\item $6\frac{2}{5}\times \frac{7}{9}$
	\item $\frac{3}{2}\times 5\frac{1}{3}$
	\item $\frac{5}{6}\times 2\frac{3}{7}$
	\item $3\frac{2}{5}\times \frac{4}{7}$
	\item $2\frac{3}{5}\times 3$
	\item $3\frac{4}{7}\times \frac{3}{5}$
	\item $\frac{2}{3}\times \rule{0.5cm}{0.1pt}=\frac{10}{30}$
	\item $\frac{3}{5}\times \rule{0.5cm}{0.1pt}=\frac{24}{75}$
	\item $7 \div \frac{2}{5}$
	\item $6 \div \frac{4}{7}$
	\item $2 \div \frac{8}{9}$
	\item $\frac{3}{5} \div \frac{1}{2}$
	\item $\frac{1}{2} \div \frac{3}{5}$
	\item $2\frac{1}{2} \div \frac{3}{5}$
	\item $5\frac{1}{6} \div \frac{9}{2}$
	\item $12 \div \frac{3}{4}$
	\item $14 \div \frac{5}{6}$
	\item $8 \div \frac{7}{3}$
	\item $4 \div \frac{8}{3}$
	\item $3 \div 2\frac{1}{3}$
	\item $5 \div 3\frac{4}{7}$
	\item $\frac{7}{3}\div 2$
	\item $\frac{4}{9}\div 5$ 
	\item $\frac{6}{13}\div 7$
	\item $4\frac{1}{3}\div 3$
	\item $3\frac{1}{2}\div 4$
	\item $4\frac{3}{7}\div 7$
	\item $\frac{2}{5} \div \frac{1}{2}$
	\item $\frac{4}{9} \div \frac{2}{3}$
	\item $\frac{3}{7} \div \frac{8}{7}$
	\item $2\frac{1}{3} \div \frac{3}{5}$
	\item $3\frac{1}{2} \div \frac{8}{3}$
	\item $\frac{2}{5} \div 1\frac{1}{2}$
	\item $3\frac{1}{5} \div 1\frac{2}{3}$
	\item $2\frac{1}{5} \div 1\frac{1}{5}$
	\item $0.2\times 6$
	\item $8\times 4.6$
	\item $2.71\times 5$
	\item $20.1\times 4$
	\item $0.05\times 7$
	\item $211.02\times 4$
	\item $2\times 0.86$
	\item $2.5\times 0.3$
	\item $0.1\times 51.7$
	\item $0.2\times 316.8$
	\item $1.3\times 3.1$
	\item $0.5\times 0.05$
	\item $11.2\times 0.15$
	\item $1.07\times 0.02$
	\item $10.05\times 1.05$
	\item $101.01\times 0.01$
	\item $100.01\times 1.1$
	\item $7.75\times 0.25$
	\item $42.8\times 0.02$
	\item $5.6\times 1.4$
	\item $0.4 \div  2$
	\item $0.35 \div  5$
	\item $2.48 \div  4$
	\item $65.4 \div  6$
	\item $651.2 \div  4$
	\item $14.49 \div  7$
	\item $3.96  \div  4$
	\item $0.80 \div  5$
	\item $7 \div 3.5$
	\item $ 36\div 0.2$
	\item $3.25 \div 0.5$
	\item $ 30.94\div 0.7$
	\item $ 0.5\div 0.25$
	\item $ 7.75\div 0.25$
	\item $ 76.5\div 0.15$
	\item $ 37.8\div 1.4$
	\item $ 2.73\div 1.3$
	\end{multicols}
\end{enumerate}
Find
\begin{enumerate}[label=\thesection.\arabic*,ref=\thesection.\theenumi,resume*,itemsep=1ex]
	\begin{multicols}{2}
	\item $\frac{1}{2}$ of
		\begin{enumerate}
			\item $2\frac{3}{4}$
			\item $4\frac{2}{9}$
		\end{enumerate}
	\item $\frac{5}{8}$ of
		\begin{enumerate}
			\item $3\frac{5}{6}$
			\item $9\frac{2}{3}$
		\end{enumerate}
	\end{multicols}
\end{enumerate}
Find
\begin{enumerate}[label=\thesection.\arabic*,ref=\thesection.\theenumi,resume*,itemsep=1ex]
	\begin{multicols}{2}
	\item $\frac{1}{4}$ of
		\begin{enumerate}
			\item $\frac{1}{4}$
			\item $\frac{3}{5}$
			\item $\frac{4}{3}$
		\end{enumerate}
	\item $\frac{1}{7}$ of
		\begin{enumerate}
			\item $\frac{2}{9}$
			\item $\frac{6}{5}$
			\item $\frac{3}{10}$
		\end{enumerate}
	\end{multicols}
\end{enumerate}
\begin{enumerate}[label=\thesection.\arabic*,ref=\thesection.\theenumi,resume*]
	\item  In a class of 40 students $\frac{1}{5}$ of the total number of students like to study English, 
$\frac{2}{5}$ of the total number like to study Mathematics and the remaining students like to study Science.
\begin{enumerate}
	\item How many students like to study English?
	\item How many students like to study Mathematics?
	\item How many students like to study Science?
\end{enumerate}
\item Vidya and Pratap went for a picnic.  Their mother gave them a water bottle that contained 5 litres of water.  Vidya consumed $\frac{2}{5}$ of the water.  Pratap consumed the remaining water.
	\begin{enumerate}
		\item How much water did Vidya drink?
		\item What fraction of the total quantity of water did Pratap drink?
	\end{enumerate}
\item Shaili plants 4 saplings in a row, in her garden.  The distance between two adjacent saplings is $\frac{3}{4}m$. Find the distance between the first and the last sapling.
\item Lipika reads a book for $1\frac{3}{4}$ hours everyday.  She reads the entire book in 6 days.  How many hours in all were required by her to read the book.
\item A car runs $16km$ using 1 litre of petrol.  How much distance will it cover using $2\frac{3}{4}$ litres of petrol.
\item The side of an equilateral triangle is $3.5cm$.  Find its perimeter.  
\item The length of a rectangle is $7.1cm$ and its breadth is $2.5cm$.  What is its area?
\item Find the area of a rectangle whose length is $5.7cm$ and breadth is $3cm$. 
\item A two wheeler covers a distance of $55.3km$ in one litre of petrol.  How much will it cover in 10 litres of petrol?
\item Savita was preparing a design to decorate her classroom.  She needed a few coloured strips of paper of length $1.9cm$ each.  She had a strip of coloured paper of length $9.5cm$.  How many pieces of the required length will she get out of this strip? 
\item Each side of a regular polygon is $2.5cm$ in length.  The perimeter of the polygon is $12.5cm$.  How many sides does the polygon have?
\item A car covers a distance of $89.1km$ in $2.2$ hours.  What is the average distance covered by it in 1 hour?
\item A vehicle covers a distance of $43.2km$ in in $2.4$ litres of petrol.  How much will it cover in one litre of petrol?
\end{enumerate}

\subsection{NCERT}
Find
\begin{enumerate}[label=\thesection.\arabic*,ref=\thesection.\theenumi,itemsep=1ex]
	\begin{multicols}{4}
%
	\item $\frac{2}{7}\times 3$
	\item $\frac{9}{7}\times 6$
	\item $\frac{1}{8}\times 3$
	\item $\frac{13}{11}\times 6$
	\item $\frac{2}{5}\times 2$
	\item $3\times 5\frac{1}{5}$
	\item $5\times 6\frac{3}{4}$
	\item $7\times 2\frac{1}{4}$
	\item $4\times 6\frac{1}{3}$
	\item $6\times 3\frac{1}{4}$
	\item $8\times 3\frac{2}{5}$
	\item $\frac{1}{2}\times \frac{1}{7}$
	\item $\frac{1}{5}\times \frac{1}{7}$
	\item $\frac{1}{3}\times \frac{4}{5}$
	\item $\frac{2}{3}\times \frac{1}{5}$
	\item $\frac{8}{3}\times \frac{4}{7}$
	\item $\frac{3}{4}\times \frac{2}{3}$
	\item $\frac{2}{3}\times 2\frac{2}{3}$
	\item $\frac{2}{7}\times \frac{7}{9}$
	\item $\frac{3}{8}\times \frac{6}{4}$
	\item $\frac{9}{5}\times \frac{3}{5}$
	\item $\frac{1}{3}\times \frac{15}{8}$
	\item $\frac{11}{2}\times \frac{3}{10}$
	\item $\frac{4}{5}\times \frac{12}{7}$
	\item $\frac{2}{5}\times 5\frac{1}{4}$
	\item $6\frac{2}{5}\times \frac{7}{9}$
	\item $\frac{3}{2}\times 5\frac{1}{3}$
	\item $\frac{5}{6}\times 2\frac{3}{7}$
	\item $3\frac{2}{5}\times \frac{4}{7}$
	\item $2\frac{3}{5}\times 3$
	\item $3\frac{4}{7}\times \frac{3}{5}$
	\item $\frac{2}{3}\times \rule{0.5cm}{0.1pt}=\frac{10}{30}$
	\item $\frac{3}{5}\times \rule{0.5cm}{0.1pt}=\frac{24}{75}$
	\item $7 \div \frac{2}{5}$
	\item $6 \div \frac{4}{7}$
	\item $2 \div \frac{8}{9}$
	\item $\frac{3}{5} \div \frac{1}{2}$
	\item $\frac{1}{2} \div \frac{3}{5}$
	\item $2\frac{1}{2} \div \frac{3}{5}$
	\item $5\frac{1}{6} \div \frac{9}{2}$
	\item $12 \div \frac{3}{4}$
	\item $14 \div \frac{5}{6}$
	\item $8 \div \frac{7}{3}$
	\item $4 \div \frac{8}{3}$
	\item $3 \div 2\frac{1}{3}$
	\item $5 \div 3\frac{4}{7}$
	\item $\frac{7}{3}\div 2$
	\item $\frac{4}{9}\div 5$ 
	\item $\frac{6}{13}\div 7$
	\item $4\frac{1}{3}\div 3$
	\item $3\frac{1}{2}\div 4$
	\item $4\frac{3}{7}\div 7$
	\item $\frac{2}{5} \div \frac{1}{2}$
	\item $\frac{4}{9} \div \frac{2}{3}$
	\item $\frac{3}{7} \div \frac{8}{7}$
	\item $2\frac{1}{3} \div \frac{3}{5}$
	\item $3\frac{1}{2} \div \frac{8}{3}$
	\item $\frac{2}{5} \div 1\frac{1}{2}$
	\item $3\frac{1}{5} \div 1\frac{2}{3}$
	\item $2\frac{1}{5} \div 1\frac{1}{5}$
	\item $0.2\times 6$
	\item $8\times 4.6$
	\item $2.71\times 5$
	\item $20.1\times 4$
	\item $0.05\times 7$
	\item $211.02\times 4$
	\item $2\times 0.86$
	\item $2.5\times 0.3$
	\item $0.1\times 51.7$
	\item $0.2\times 316.8$
	\item $1.3\times 3.1$
	\item $0.5\times 0.05$
	\item $11.2\times 0.15$
	\item $1.07\times 0.02$
	\item $10.05\times 1.05$
	\item $101.01\times 0.01$
	\item $100.01\times 1.1$
	\item $7.75\times 0.25$
	\item $42.8\times 0.02$
	\item $5.6\times 1.4$
	\item $0.4 \div  2$
	\item $0.35 \div  5$
	\item $2.48 \div  4$
	\item $65.4 \div  6$
	\item $651.2 \div  4$
	\item $14.49 \div  7$
	\item $3.96  \div  4$
	\item $0.80 \div  5$
	\item $7 \div 3.5$
	\item $ 36\div 0.2$
	\item $3.25 \div 0.5$
	\item $ 30.94\div 0.7$
	\item $ 0.5\div 0.25$
	\item $ 7.75\div 0.25$
	\item $ 76.5\div 0.15$
	\item $ 37.8\div 1.4$
	\item $ 2.73\div 1.3$
	\end{multicols}
\end{enumerate}
Find
\begin{enumerate}[label=\thesection.\arabic*,ref=\thesection.\theenumi,resume*,itemsep=1ex]
	\begin{multicols}{2}
	\item $\frac{1}{2}$ of
		\begin{enumerate}
			\item $2\frac{3}{4}$
			\item $4\frac{2}{9}$
		\end{enumerate}
	\item $\frac{5}{8}$ of
		\begin{enumerate}
			\item $3\frac{5}{6}$
			\item $9\frac{2}{3}$
		\end{enumerate}
	\end{multicols}
\end{enumerate}
Find
\begin{enumerate}[label=\thesection.\arabic*,ref=\thesection.\theenumi,resume*,itemsep=1ex]
	\begin{multicols}{2}
	\item $\frac{1}{4}$ of
		\begin{enumerate}
			\item $\frac{1}{4}$
			\item $\frac{3}{5}$
			\item $\frac{4}{3}$
		\end{enumerate}
	\item $\frac{1}{7}$ of
		\begin{enumerate}
			\item $\frac{2}{9}$
			\item $\frac{6}{5}$
			\item $\frac{3}{10}$
		\end{enumerate}
	\end{multicols}
\end{enumerate}
\begin{enumerate}[label=\thesection.\arabic*,ref=\thesection.\theenumi,resume*]
	\item  In a class of 40 students $\frac{1}{5}$ of the total number of students like to study English, 
$\frac{2}{5}$ of the total number like to study Mathematics and the remaining students like to study Science.
\begin{enumerate}
	\item How many students like to study English?
	\item How many students like to study Mathematics?
	\item How many students like to study Science?
\end{enumerate}
\item Vidya and Pratap went for a picnic.  Their mother gave them a water bottle that contained 5 litres of water.  Vidya consumed $\frac{2}{5}$ of the water.  Pratap consumed the remaining water.
	\begin{enumerate}
		\item How much water did Vidya drink?
		\item What fraction of the total quantity of water did Pratap drink?
	\end{enumerate}
\item Shaili plants 4 saplings in a row, in her garden.  The distance between two adjacent saplings is $\frac{3}{4}m$. Find the distance between the first and the last sapling.
\item Lipika reads a book for $1\frac{3}{4}$ hours everyday.  She reads the entire book in 6 days.  How many hours in all were required by her to read the book.
\item A car runs $16km$ using 1 litre of petrol.  How much distance will it cover using $2\frac{3}{4}$ litres of petrol.
\item The side of an equilateral triangle is $3.5cm$.  Find its perimeter.  
\item The length of a rectangle is $7.1cm$ and its breadth is $2.5cm$.  What is its area?
\item Find the area of a rectangle whose length is $5.7cm$ and breadth is $3cm$. 
\item A two wheeler covers a distance of $55.3km$ in one litre of petrol.  How much will it cover in 10 litres of petrol?
\item Savita was preparing a design to decorate her classroom.  She needed a few coloured strips of paper of length $1.9cm$ each.  She had a strip of coloured paper of length $9.5cm$.  How many pieces of the required length will she get out of this strip? 
\item Each side of a regular polygon is $2.5cm$ in length.  The perimeter of the polygon is $12.5cm$.  How many sides does the polygon have?
\item A car covers a distance of $89.1km$ in $2.2$ hours.  What is the average distance covered by it in 1 hour?
\item A vehicle covers a distance of $43.2km$ in in $2.4$ litres of petrol.  How much will it cover in one litre of petrol?
\end{enumerate}

\section{Programming}
\subsection{Formulae}
In a quiz, team A scored $a_1 = -40, a_2=10, a_3=0$ and team B scored $b_1=10, b_2=0, b_3=-40$ in three successive rounds.
\begin{enumerate}[label=\thesection.\arabic*, ref=\thesubsection.\theenumi]
\item  If the total scores are 
	\begin{align}
		a &= a_1+a_2+a_3
		\\
		b &= b_1+b_2+b_3
	\end{align}
	which team scored more? 
	\\
	\solution 
	\lstinputlisting{codes/ifelse.c}
\item Write a function to compare the final scores.  Check for the cases when $a = -40, b = -40; a = 30, b = 20; a = -20, b = -10$.
	\\
	\solution 
	\lstinputlisting{codes/func.c}
\item Use arrays and a for loop to evaluate 
	\begin{align}
		a &= \sum_{i=0}^{2}a_i
		\\
		b &= \sum_{i=0}^{2}b_i
	\end{align}
	\\
	\solution 
	\lstinputlisting{codes/loop.c}
\item Revise the above code using only functions.
	\\
	\solution 
	\lstinputlisting{codes/loopfunc.c}
\item Use files for the input data.
	\\
	\solution 
	\lstinputlisting{codes/files.c}
\item Revise the files program using pointer arrays
	\\
	\solution 
	\lstinputlisting{codes/pointer.c}
\item Revise the files program using only functions
	\\
	\solution 
	\lstinputlisting{codes/filesfunc.c}
\end{enumerate}

\subsection{NCERT}
In a quiz, team A scored $a_1 = -40, a_2=10, a_3=0$ and team B scored $b_1=10, b_2=0, b_3=-40$ in three successive rounds.
\begin{enumerate}[label=\thesection.\arabic*, ref=\thesubsection.\theenumi]
\item  If the total scores are 
	\begin{align}
		a &= a_1+a_2+a_3
		\\
		b &= b_1+b_2+b_3
	\end{align}
	which team scored more? 
	\\
	\solution 
	\lstinputlisting{codes/ifelse.c}
\item Write a function to compare the final scores.  Check for the cases when $a = -40, b = -40; a = 30, b = 20; a = -20, b = -10$.
	\\
	\solution 
	\lstinputlisting{codes/func.c}
\item Use arrays and a for loop to evaluate 
	\begin{align}
		a &= \sum_{i=0}^{2}a_i
		\\
		b &= \sum_{i=0}^{2}b_i
	\end{align}
	\\
	\solution 
	\lstinputlisting{codes/loop.c}
\item Revise the above code using only functions.
	\\
	\solution 
	\lstinputlisting{codes/loopfunc.c}
\item Use files for the input data.
	\\
	\solution 
	\lstinputlisting{codes/files.c}
\item Revise the files program using pointer arrays
	\\
	\solution 
	\lstinputlisting{codes/pointer.c}
\item Revise the files program using only functions
	\\
	\solution 
	\lstinputlisting{codes/filesfunc.c}
\end{enumerate}

\section{Data Handling}
\subsection{Formulae}
\begin{enumerate}[label=\thesection.\arabic*, ref=\thesection.\theenumi]
\item Find
	\begin{enumerate}
	\begin{multicols}{3}
	\item $2.7\times 4$
	\item $1.8\times 1.2$
	\item $2.3\times 4.35$
	\end{multicols}
	\end{enumerate}
and arrange the products in descending order.
\item Find the average of $4.2, 3.8$ and $7.6$.
\item Ashish studies for 4 hours, 5 hours and and 3 hours respectively on three consecutive days.  How many hours does he study daily on an average?
\item A batsman scored the following number of runs in 6 innings.  
	$$36, 35, 50, 46, 60, 55$$
	Calculate the mean runs scored by him in an inning.
\item The ages in years of 10 teachers of a school are
	$$32, 41, 28, 54, 35, 26, 23, 33, 38, 40$$
	\begin{enumerate}
		\item What is the age of the oldest teacher and that of the youngest teacher?
		\item What is the range of the ages of the teachers?
		\item What is the mean age of these teachers?
	\end{enumerate}
\item Organize the following marks in a class assessment, in tabular form with columns as marks and frequency.
	\begin{enumerate}
		\item Which number is the highest?
		\item Which number is the lowest?
		\item What is the range of the data?
		\item Find the arithmetic mean.
	\end{enumerate}
\item A cricketer scores the following runs in eight innings.
	$$58, 76, 40, 35, 46, 45, 0, 100$$
	Find the mean score.
\item Generate the following table using a C program
	\begin{figure}[H]
  \centering
  \includegraphics[width=\columnwidth]{figs/data.jpg}
  \caption{}
  \label{fig:data}
\end{figure}
and answer the following questions.
\begin{enumerate}
	\item Find the mean to determine $A's$ average number of points scored per game.
	\item Who is the best performer?
\end{enumerate}
\item The marks out of 100 obtained by a group of students in a science test are 85, 76, 90, 85, 39, 48, 56, 95, 81 and 75.  Find the 
	\begin{enumerate}
		\item Highest and lowest marks obtained by the students.
		\item Range of marks obtained.
		\item  Mean marks obtained by the group.
	\end{enumerate}
\item The enrolment in a school during six consecutive years was as follows  
	$$1555, 1670, 1750, 2013, 2540, 2820$$
	Find the mean enrolment of the school for this period.
\item The rainfall (in mm) in a city on 7 days a week was recorded as in 
  \tabref{fig:data2}.  Generate this table using a C program.
	\begin{figure}[H]
  \centering
  \includegraphics[width=\columnwidth]{figs/data.jpg}
  \caption{}
  \label{fig:data2}
\end{figure}
\item Find the range of the rainfall in the given data.
\item Find the mean rainfall for the week.
\item On how many days was the rainfall less than the mean rainfall 
\item The height of 10 girls was measured in cm and result was as follows
	$$135, 150, 139, 128, 151, 132, 146, 149, 143, 141.$$
	\begin{enumerate}
		\item What is the height of the tallest girl?
		\item What is the height of the shortest girl?
		\item What is the range of the data?
		\item What is the mean height of the girls?
		\item How many girls have heights more than the mean height?
	\end{enumerate}
\item To find out the weekly demand for different sizes of shirt, a shopkeeper kept records of sales of sizes as shown in 
  \eqref{fig:mode}.  This is the record for a week.  Find the mode of the data.
	\begin{figure}[H]
  \centering
  \includegraphics[width=\columnwidth]{figs/mode.jpg}
  \caption{}
  \label{fig:mode}
\end{figure}
\item Find the mode of the given set of numbers
	$$1,1,1,2,2,2,2,3,4,4$$.
\item Following are the margins of victory in the football matches of a league.  Find the mode of this data.
	\begin{gather}
	1,3,2,5,1,4,6,2,5,2,2,2,4,1,2,3,1,1,2,3,2,6,4,3,2,
	\\
	1,1,4,2,1,5,3,3,2,3,2,42,1,2.
	\end{gather}
\end{enumerate}

\subsection{NCERT}
\begin{enumerate}[label=\thesection.\arabic*, ref=\thesection.\theenumi]
\item Find
	\begin{enumerate}
	\begin{multicols}{3}
	\item $2.7\times 4$
	\item $1.8\times 1.2$
	\item $2.3\times 4.35$
	\end{multicols}
	\end{enumerate}
and arrange the products in descending order.
\item Find the average of $4.2, 3.8$ and $7.6$.
\item Ashish studies for 4 hours, 5 hours and and 3 hours respectively on three consecutive days.  How many hours does he study daily on an average?
\item A batsman scored the following number of runs in 6 innings.  
	$$36, 35, 50, 46, 60, 55$$
	Calculate the mean runs scored by him in an inning.
\item The ages in years of 10 teachers of a school are
	$$32, 41, 28, 54, 35, 26, 23, 33, 38, 40$$
	\begin{enumerate}
		\item What is the age of the oldest teacher and that of the youngest teacher?
		\item What is the range of the ages of the teachers?
		\item What is the mean age of these teachers?
	\end{enumerate}
\item Organize the following marks in a class assessment, in tabular form with columns as marks and frequency.
	\begin{enumerate}
		\item Which number is the highest?
		\item Which number is the lowest?
		\item What is the range of the data?
		\item Find the arithmetic mean.
	\end{enumerate}
\item A cricketer scores the following runs in eight innings.
	$$58, 76, 40, 35, 46, 45, 0, 100$$
	Find the mean score.
\item Generate the following table using a C program
	\begin{figure}[H]
  \centering
  \includegraphics[width=\columnwidth]{figs/data.jpg}
  \caption{}
  \label{fig:data}
\end{figure}
and answer the following questions.
\begin{enumerate}
	\item Find the mean to determine $A's$ average number of points scored per game.
	\item Who is the best performer?
\end{enumerate}
\item The marks out of 100 obtained by a group of students in a science test are 85, 76, 90, 85, 39, 48, 56, 95, 81 and 75.  Find the 
	\begin{enumerate}
		\item Highest and lowest marks obtained by the students.
		\item Range of marks obtained.
		\item  Mean marks obtained by the group.
	\end{enumerate}
\item The enrolment in a school during six consecutive years was as follows  
	$$1555, 1670, 1750, 2013, 2540, 2820$$
	Find the mean enrolment of the school for this period.
\item The rainfall (in mm) in a city on 7 days a week was recorded as in 
  \tabref{fig:data2}.  Generate this table using a C program.
	\begin{figure}[H]
  \centering
  \includegraphics[width=\columnwidth]{figs/data.jpg}
  \caption{}
  \label{fig:data2}
\end{figure}
\item Find the range of the rainfall in the given data.
\item Find the mean rainfall for the week.
\item On how many days was the rainfall less than the mean rainfall 
\item The height of 10 girls was measured in cm and result was as follows
	$$135, 150, 139, 128, 151, 132, 146, 149, 143, 141.$$
	\begin{enumerate}
		\item What is the height of the tallest girl?
		\item What is the height of the shortest girl?
		\item What is the range of the data?
		\item What is the mean height of the girls?
		\item How many girls have heights more than the mean height?
	\end{enumerate}
\item To find out the weekly demand for different sizes of shirt, a shopkeeper kept records of sales of sizes as shown in 
  \eqref{fig:mode}.  This is the record for a week.  Find the mode of the data.
	\begin{figure}[H]
  \centering
  \includegraphics[width=\columnwidth]{figs/mode.jpg}
  \caption{}
  \label{fig:mode}
\end{figure}
\item Find the mode of the given set of numbers
	$$1,1,1,2,2,2,2,3,4,4$$.
\item Following are the margins of victory in the football matches of a league.  Find the mode of this data.
	\begin{gather}
	1,3,2,5,1,4,6,2,5,2,2,2,4,1,2,3,1,1,2,3,2,6,4,3,2,
	\\
	1,1,4,2,1,5,3,3,2,3,2,42,1,2.
	\end{gather}
\end{enumerate}

\section{Math Library}
\begin{enumerate}[label=\thesection.\arabic*, ref=\thesection.\theenumi]
\item Determine whether the triangle whose lengths of sides are $3 cm, 4 cm, 5 cm$ is a right-angled triangle.
\item $\triangle ABC$ is right-angled at $C$. If $AC = 5 cm$ and $BC = 12 cm$ find the length of $AB$.
\item $PQR$ is a triangle, right-angled at $P$. If $PQ = 10cm$ and $PR = 24 cm$, find $QR$.
\item $ABC$ is a triangle, right-angled at $C$. If $AB = 25 cm$ and $AC = 7 cm$, find $BC$.
\item A $15 m$ long ladder reached a window $12 m$ high from the ground on placing it against a wall at a distance $a$. Find the distance of the foot of the ladder from the wall.
\item  Which of the following can be the sides of a right triangle? 
\begin{enumerate}
	\item $2.5 cm,6.5 cm, 6 cm.$ 
	\item $ 2 cm, 2 cm, 5 cm.$ 
	\item $ 1.5 cm, 2cm, 2.5 cm.$
\end{enumerate}
\item A tree is broken at a height of $5 m$ from the ground and its top touches the ground at a distance of $12 m$ from the base of the tree. Find the original height of the tree.
\item Find the perimeter of the rectangle whose length is $40 cm$ and a diagonal is $41 cm$. 
\item The diagonals of a rhombus measure $16 cm$ and $30 cm$. Find its perimeter.
\item Find the values of the following expressions for $x = 2$. 
	\begin{enumerate}
\item $x + 4$
\item  $4x – 3$ 
\item  $19-5x^2$
\item  $100 – 10x^3$
	\end{enumerate}
\item Find the value of the following expressions when $n = – 2.$ 
	\begin{enumerate}
\item $5n – 2$
\item  $5n^2 + 5n – 2 $
\item  $n^3 + 5n^2 + 5n – 2$
	\end{enumerate}
\item Find the value of the following expressions for $a = 3, b = 2$. 
	\begin{enumerate}
\item $a + b$				
\item $ 7a – 4b $
\item $ a^2+2ab+b^2$
\item $ a^3 – b^3$
\end{enumerate}
\item If $m = 2$, find the value of
	\begin{enumerate}
\item $ m – 2$
\item $ 3m – 5 $
\item $ 9 – 5m $
\item $ 3m^2 – 2m – 7 $
\item $ \frac{5m^4}{ 2}$
\end{enumerate}
\item  If $p = – 2$, find the value of: 
	\begin{enumerate}
\item $4p + 7$
\item  $– 3p^2 + 4p + 7 $
\item  $– 2p^3 – 3p^2 + 4p + 7$
\end{enumerate}
\item  Find the value of the following expressions, when $x = –1$ 
	\begin{enumerate}
\item $ 2x – 7$
\item $ – x + 2$ 
\item $ 2x^2 – x – 2$
\item $ x^2 + 2x +1$
\end{enumerate}
\item  If $a = 2, b = – 2$, find the value of
	\begin{enumerate}
\item  $ a^2 + b^2$
\item  $a^2 + ab + b^2 $
\item  $a^2 – b^2$
\end{enumerate}
\item  When $a = 0, b = – 1$, find the value of the given expressions
	\begin{enumerate}
\item $2a + 2b$
\item $2a^2 + b^2 + 1 $
\item $2a^2b + 2ab^2 + ab $
\item $a^2 + ab + 2$
\end{enumerate}
\item  Simplify the expressions and find the value if $x$ is equal to 2 
	\begin{enumerate}
\item $ x + 7 + 4 (x – 5)$
\item $ 3 (x + 2) + 5x – 7 $
\item $ 6x + 5 (x – 2) $
\item $ 4(2x – 1) + 3x + 11$
\end{enumerate}
\item  Simplify these expressions and find their values if $x = 3, a = – 1, b = – 2$. 
	\begin{enumerate}
\item $ 2x +4 $
\item $6 - 4x$
\item $6 – 5a $
\item $6 – 8b $
\item $3a – 2b-9 $
\end{enumerate}
\item If $z = 10$, find the value of $z^3 – 3(z – 10)$. 
\item  If $p = – 10$, find the value of $p^2 – 2p – 100$
\item  Simplify the expression and find its value when $a = 5$ and $b = – 3$
	$$ 2a^2 + ab + 3 $$
\item What is the circumference of a circle of diameter 10 cm?
\item What is the circumference of a circular disc of radius 14 cm?
\item The radius of a circular pipe is 10 cm. What length of a tape is required to wrap once around the pipe?
\item Sudhanshu divides a circular disc of radius 7 cm in two equal parts. What is the perimeter of each semicircular shape disc?
\item Find the area of a circle of radius 30 cm?
\item Diameter of a circular garden is 9.8 m. Find its area.
\item Find the circumference of the circles with the following radius  
	\begin{enumerate}
		\item 28 mm
\item  14 cm
\end{enumerate}
\item  Find the area of the following circles, given that the radius is
	\begin{enumerate}
\item 14 mm 
\item 5 cm
\item 21 cm 
\item  diameter = 49 m
\end{enumerate}
\item If the circumference of a circular sheet is 154 m, find its radius. Also find the area of the sheet. 
\item A gardener wants to fence a circular garden of diameter 21m. Find the length of the rope he needs to purchase, if he makes 2 rounds of fence. Also find the cost of the rope, if it costs \rupee 4 per meter. 
\item  From a circular sheet of radius 4 cm, a circle of radius 3 cm is removed. Find the area of the remaining sheet. 
\item Seema wants to put a lace on the edge of a circular table cover of diameter 1.5 m. Find the length of the lace required and also find its cost if one meter of the lace costs
\rupee 15. 
\item Find the cost of polishing a circular table-top of diameter 1.6 m, if the rate of polishing is \rupee $15/m^2$. 
\item Shalya took a wire of length 44 cm and bent it into the shape of a circle. Find the radius of that circle. Also find its area. If the same wire is bent into the shape of a square, what will be the length of each of its sides? Which figure encloses more
area, the circle or the square? 
\item From a circular card sheet of radius 14 cm, two circles of radius 3.5 cm and a rectangle of length 3 cm and breadth 1cm are removed. 
 Find the area of the remaining sheet. 
\item A circle of radius 2 cm is cut out from a square piece of an aluminium sheet of side 6 cm. What is the area of the left over aluminium sheet? 
\item  The circumference of a circle is 31.4 cm. Find the radius and the area of the circle. 
\item A circular flower bed is surrounded by a path 4 m wide. The diameter of the flower bed is 66 m. What is the area of this path? 
\item A circular flower garden has an area of $314 m^2$. A sprinkler at the centre of the garden can cover an area that has a radius of 12 m. Will the sprinkler water the entire garden? 
\item Find the circumference of the inner and the outer circles, shown in the adjoining figure? 
\item How many times a wheel of radius 28 cm must rotate to go 352 m? 
\item The minute hand of a circular clock is 15 cm.
	How far does the tip of the minute hand move in 1 hour? 
\item The two sides of the parallelogram ABCD are 6 cm and 4 cm. The height corresponding to the base CD is 3 cm. Find the
	\begin{enumerate}
\item 	area of the parallelogram. 
\item the height corresponding to the base AD.
\end{enumerate}
\item 		\end{enumerate}

\section{Random Numbers}
\input{ncert/rv.tex}
\iffalse
\subsection{NCERT}
Verify
\begin{enumerate}[label=\thesubsection.\arabic*.,ref=\thesubsection.\theenumi,resume*]
	\item $x^3+y^3 = \brak{x+y}\brak{x^2-xy+y^2}$
	\item $x^3-y^3 = \brak{x-y}\brak{x^2+xy+y^2}$
	\item $x^3+y^3+z^3-3xyz = \frac{1}{2}\brak{x+y+z}\sbrak{\brak{x-y}^{2}+\brak{y-z}^{2}+\brak{z-x}^{2}}$
\end{enumerate}
Factorize each of the following
\begin{enumerate}[label=\thesubsection.\arabic*.,ref=\thesubsection.\theenumi]
	\item $8a^3+b^3+12a^2b+6ab^2$
	\item $8a^3-b^3-12a^2b+6ab^2$
	\item $27-125a^3-135a+225a^2$
	\item $64a^3-27b^3-144a^2b+108ab^2$
	\item $27p^3-\frac{1}{216}-\frac{9}{2}p^2 + \frac{p}{4}$
	\item $27y^3+125z^3$ 
	\item $64m^3-343n^3$
	\item $27x^3+y^3+z^3-9xyz$
\end{enumerate}
Find the value of each of the following 
\begin{enumerate}[label=\thesubsection.\arabic*.,ref=\thesubsection.\theenumi,resume*]
	\item $\brak{-12}^3+\brak{7}^3+\brak{5}^3$
	\item $\brak{28}^3+\brak{-15}^3+\brak{-13}^3$
\end{enumerate}
Give possible expressions for the length and breadth of each of the following rectangles, in which their areas are given
\begin{enumerate}[label=\thesubsection.\arabic*.,ref=\thesubsection.\theenumi,resume*]
	\item $25a^2-35a+12$
	\item $35a^2+13y-12$
\end{enumerate}
What are the possible expressions for the dimensions of the cuboids whose volumes are given below
\begin{enumerate}[label=\thesubsection.\arabic*.,ref=\thesubsection.\theenumi,resume*]
	\item $3x^2-12x$
	\item $12ky^2+8ky-20k$
\end{enumerate}

\section{Polynomials}
\subsection{NCERT}
\begin{enumerate}[label=\thesubsection.\arabic*, ref=\thesubsection.\theenumi,resume*]
%
\item Divide $p(x)$ by $g(x)$, where $p(x) = x + 3x^2– 1$ and $g(x) = 1 + x$.
\item Divide the polynomial $p(x) = 3x^4-4x^3-3x-1 $ by $x-1$.
\item Find the remainder obtained upon dividing $p(x) = x^3+1$ by $x+1$.
\item Find the remainder when $x^4+x^3-2x^2+x+1$ is divided by $x-1$.
\item Check whether the polynomial $q(t)=4t^3+4t^2-t-1$ is a multiple of $2t+1$.
\item Find the remainder when $x^3+3x^2+3x+1$ is divided by 
	\begin{enumerate}
		\item $x+1$
		\item $x-\frac{1}{2}$
		\item $x$
		\item $x+\pi$
		\item $5+2x$
	\end{enumerate}
\item Check whether $7+3x$ is a factor of $3x^3+7x$.
\item Find the value of $k$, if $x – 1$ is a factor of $p(x) = 4x^3+ 3x^2 - 4x + k$.
%
\item Determine which of the following polynomials has $x+1$ as a factor
	\begin{enumerate}
		\item $x^3+x^2+x+1$
		\item $x^4+x^3+x^2+x+1$
		\item $x^4+3x^3+3x^2+x+1$
		\item $x^3-x^2-\brak{2+\sqrt{2}}x+\sqrt{2}$
	\end{enumerate}
\item Factorize $x^3-23x^2+142x-120$.
\item Divide $2x^2+3x+1$ by $x+2$.
\item Divide $3x^3+x^2+2x+5$ by $1+2x+x^2$.
\item Find all the zeroes of $2x^4-3x^3-3x^2+6x-2$, if you know that two of its zeroes are $\sqrt{2}$ and $-\sqrt{2}$.
\item Find the remainder when $x^3-ax^2 +6x-a$ is divided by $x-a$.
\item Find the value of k, if x – 1 is a factor of p(x) in each of the following cases: 
\begin{enumerate}
\item $p(x) = x^2 + x + k$
\item $p(x) = kx^2-\sqrt{2}x+1$
\item $p(x) = 2x^2 + kx + \sqrt{2}$
\item $p(x) = kx^2  - 3x + k$
\end{enumerate}
\item Divide the polnyomial $p(x)$ by the polynomial $g(x)$ and find the quotient and remainder in each of the following:
\begin{enumerate}
\item $p(x) = x^3-3x^2+5x-3, g(x) = x^2-2$.
\item $p(x) = x^4-3x^2+4x+5, g(x) = x^2+1-x$.
\item $p(x) = x^4-5x+6, g(x) = 2-x^2$.
\end{enumerate}
\item Check whether the first polynomial is a factor of the second polynomial by dividing the second polynomial by the first polynomial:
\begin{enumerate}
\item $t^2-x,2t^4+3t^3-2t^2-9t-12$.
\item $x^2+3x+1, 3x^4+5x^3-7x^2+2x+2$.
\item $x^3-3x+1, x^5-4x^3+x^2+3x+1$.
\end{enumerate}
%
\item Obtain all the other zeroes of $3x^4+6x^3-2x^2-10x-5$, if two of its zeroes are $\sqrt{\frac{5}{3}}$and $-\sqrt{\frac{5}{3}}$.
\item On dividing $x^3-3x^2+x+2$ by a polynomial $g(x)$, the quotient and remainder were $x-2$ and $-2x+4$respectively.  Find $g(x)$.
\item Verify that the numbers given alongside the cubic polynomials below are their zeroes.  Also verify if the relationship between the zeroes and the coefficients in each case:
\begin{enumerate}
\item $2x^3+x^2-5x+2; \frac{1}{2}, 1, -2$
\item $x^3-4x^2+5x-2; 2, 1, 1$
\end{enumerate}
\item Find a cubic polynomial with the sum, sum of the product of its zeroes taken two at a time, and the product of its zeroes as 2, -7, -4 respectively.
\item If two zeroes of the polynomial $x^4-6x^3-26x^2+138x-35$ are $2\pm \sqrt{3}$, find the other zeroes.\item If the polynomial $x^4-6x^3+16x^2-25x+10$ is divided by another polynomial $x^2-2x+k$, the remainder comes out to be $x+a$, find $k$ and $a$.
\item Use the factor theorem to determine whether $g(x)$ is a factor of $p(x)$ in each of the following cases.
	\begin{enumerate}
		\item $p(x) = 2x^3+x^2-2x-1, g(x) = x+1$
		\item $p(x) = x^3+3x^2+3x+1, g(x) = x+2$
		\item $p(x) = x^3-4x^2+x+6, g(x) = x-3$
	\end{enumerate}
\item Factorise
	\begin{enumerate}
		\item $12x^2-7x + 1$
		\item $2x^2+7x + 3$
		\item $6x^2+5x - 6$
		\item $3x^2-x - 4$
		\item $x^3-2x^2 - x+2$
		\item $x^3-3x^2 - 9x-5$
		\item $x^3+13x^2 +32x+20$
		\item $2y^3+y^2 - 2y+1$
\end{enumerate}
\item Factorise $y^2-5y+6$ using the factor theorem.
\end{enumerate}

\section{Roots}
\subsection{NCERT}
Find the roots of the following equations graphically or otherwise.
\begin{enumerate}[label=\thesubsection.\arabic*, ref=\thesubsection.\theenumi]
	\begin{multicols}{2}
	\item $x^2-2x=0$
	\item $x^3-x^2+2=0$
	\item $x^5-x^4+3=0$
	\item $2-y^2-y^3+2y^8=0$
	\item $x-x^3=1$
	\item $5x^3+4x^2+7x=0$
	\item $4-y^2=0$
	\item $x^2+x+2 = 0$
	\item $4x^2-3x+7 = 0$
	\item $y^2+\sqrt{2} = 0$
	\item $3\sqrt{t}+t\sqrt{2} = 1$
	\item $y+\frac{2}{y} = 1$
	\item $x^2+x = 1$
	\item $y+{y}^2+4 = 0$
	\item $y(x)=5x^2-3x+7 = 0$.  Find $y(1).$
	\item $3y^3-4y+\sqrt{11}=0$.  Find $y(2).$
	\item $4t^4+5t^3-t^2+6=0$
	\item $5x^3-2x^2+3x-2=0$.  Find $y(1), y(0)$ and $y(-1)$.
	\item $5x-4x^2+3=0$.  Find $y(2), y(0)$ and $y(-1)$.
	\item $\brak{x-2}^2+1 = 2x-3$
	\item $x\brak{2x+3} = x^2+1$
	\item $\brak{x+2}^3 = x^3-4$
	\item $\brak{x+1}^2 = 2\brak{x-3}$
	\item $x^2-2x = -2\brak{3-x}$
	\item $\brak{2x-1}\brak{x-3} = \brak{x+5}\brak{x-1}$
	\item $\brak{x+2}^{3} = 2x\brak{x^2-1}$
	\item ${x}^{3}-4x^2-x+1 = \brak{x-2}^3$
	\item $2{x}^{2}-5x+3 = 0$
	\item $6{x}^{2}-x-2 = 0$
	\item $3{x}^{2}-2\sqrt{6}x+2 = 0$
	\item ${x}^{2}-3x-10 = 0$
	\item $2{x}^{2}+x-6 = 0$
	\item $\sqrt{2}{x}^{2}+7x+5\sqrt{2} = 0$
	\item $2{x}^{2}-x+\frac{1}{8} = 0$
	\item $100{x}^{2}-20x+1 = 0$
	\item $5{x}^{2}-6x-2 = 0$
	\item $4{x}^{2}+3x+5 = 0$
	\item $3{x}^{2}-5x+2 = 0$
	\item ${x}^{2}+4x+5 = 0$
	\item $2{x}^{2}-2\sqrt{2}x+1 = 0$
	\item $x+\frac{1}{x}=3, x\neq{0}$
\item $\frac{1}{x}-\frac{1}{x-2} = 3, x\neq 0,2$
\item $3x^2-2x+\frac{1}{3} = 0$. 
\item $x^2-4x+3 = 0$.
\item $2x^2-4x+3 = 0$.
\item $x-\frac{1}{x}=3, x\neq{0}$
\item
$\frac{1}{x+4}-\frac{1}{x-7}=\frac{11}{30}, x\neq{-4,7}$
\item $2x^2-3x+5=0$
\item $3x^2-4 \sqrt 3x+4=0$
\item $2x^2-6x+3=0$
	\end{multicols}
\end{enumerate}
Find $p(0), p(1)$ and $p(2)$ for each of the following polynomials.
\begin{enumerate}[label=\thesubsection.\arabic*, ref=\thesubsection.\theenumi,resume*]
	\begin{multicols}{2}
	\item $p(y)=y^2-y+1$
	\item $p(t)=2+t+2t^2-t^3$
	\item $p(x)=x^3$
	\item $p(x)=\brak{y-1}\brak{y+1}$
	\end{multicols}
\end{enumerate}
Find the values of $k$ for each of the following quadratic equations, so that they have two equal roots
\begin{enumerate}[label=\thesubsection.\arabic*, ref=\thesubsection.\theenumi,resume*]
\item 	$2x^2+kx+3 = 0$
\item 	$kx\brak{x-2}+6= 0$
\end{enumerate}
Verify whether the following are zeroes of the polynomial, indicated against them.
\begin{enumerate}[label=\thesubsection.\arabic*, ref=\thesubsection.\theenumi,resume*]
	\item $p(x) = x^2-1, \quad x=-1, 1$.
	\item $p(x) = \brak{x-2}\brak{x+1}, \quad x=-1, 2$.
	\item $p(x) = 3x^2-1, x=-\frac{1}{\sqrt{3}},\frac{2}{\sqrt{3}}$.
\end{enumerate}

\section{Quadratic Equations}
\subsection{NCERT}
\begin{enumerate}[label=\thesubsection.\arabic*,ref=\thesubsection.\theenumi]
%
	\item Janak and Jivanti together have 45 marbles. Both of them lost 5 marbles each, and the product of the number of marbles they now have is 124. We would like to find out how many marbles they had to start with.
\item  A cottage industry produces a certain number of toys in a day. The cost of production of each toy (in rupees) was found to be 55 minus the number of toys produced in a day. On a particular day, the total cost of production was \rupee 750. We would like to find out the number of toys produced on that day.
\item The product of Sunita’s age (in years) two years ago and her age four years from now is one more than twice her present age. What is her present age?
\item Find two consecutive odd positive integers, sum of whose squares is 290.
\item A motor boat whose speed is 18 km/h in still water takes 1 hour more to go 24 km upstream than to return downstream to the same spot. Find the speed of the stream.
%
\item The product of two consecutive positive integers is 306. We need to find the integers.
\item Rohan’s mother is 26 years older than him. The product of their ages (in years) 3 years from now will be 360. We would like to find Rohan’s present age.
\item A train travels a distance of 480 km at a uniform speed. If the speed had been 8 km/h less, then it would have taken 3 hours more to cover the same distance. We need to find the speed of the train.
\item Find two numbers whose sum is 27 and product is 182. 
\item  Find two consecutive positive integers, sum of whose squares is 365. 
\item  A cottage industry produces a certain number of pottery articles in a day. It was observed on a particular day that the cost of production of each article (in rupees) was 3 more than twice the number of articles produced on that day. If the total cost of production on that day was \rupee 90, find the number of articles produced and the cost of each article.
\item The sum of the reciprocals of Raman’s ages, (in years) 3 years ago and 5 years from now is $\frac{1}{3}$.  Find his present age.
\item In a class test, the sum of Shefali’s marks in Mathematics and English is 30. Had she got 2 marks more in Mathematics and 3 marks less in English, the product of their marks would have been 210. Find her marks in the two subjects.
\item The difference of squares of two numbers is 180. The square of the smaller number is 8 times the larger number. Find the two numbers.
\item A train travels 360 km at a uniform speed. If the speed had been 5 km/h more, it would have taken 1 hour less for the same journey. Find the speed of the train.
\item Two water taps together can fill a tank in 9$\frac{3}{ 8}$
hours. The tap of larger diameter takes 10
hours less than the smaller one to fill the tank separately. Find the time in which each tap can separately fill the tank.
\item An express train takes 1 hour less than a passenger train to travel 132 km between Mysore and Bangalore (without taking into consideration the time they stop at intermediate stations). If the average speed of the express train is 11km/h more than that of the passenger train, find the average speed of the two trains.
\item Sum of the areas of two squares is 468 $m^22$ find the sides of the two squares.
\item Is the following situation possible? If so, determine their present ages. The sum of the ages of two friends is 20 years. Four years ago, the product of their ages in years was 48.
%
\item The area of a rectangular plot is $528m^2$.  The length of the plot is one more than twice the breadth. We need to find the length and breadth of the plot. 
\item A temple courtyard has a carpet area of $300m^2$ with its length one metre more than twice its breadth.  What should be the length and breadth of the hall.
\item The altitude of a right triangle is $7cm$ less than its base. If the hypotenuse is $13 cm$, find the other two sides. 
\item A rectangular park is to be designed whose breadth is $3$ m less than its length. Its area is to be $4$ square metres than the area of a park that has already been made in the shape of a isoceles triangle with its base as  the breadth of the rectangular park and of altitude $12$ m. Find its length and breadth.
\item The diagonal of a rectangular field is $60$ metres more than the shorter side. If the longer side is $30$ metres more than the shorter side, find the sides of the field.
\item The difference of squares of two numbers is $180$. The square of the smaller number is $8$ times the larger number. Find the two numbers.
\item A train travels $360$ km at a uniform speed. If the speed had been $5$ km/hr more, it would have taken $1$ hour less for the same journey. Find the speed of the train.
\item Two water taps together can fill a tank in $9\frac{3}{8}$ hours. The tap of larger diameter takes $10$ hours. The tap of larger diameter takes $10$ hours less than the smaller one to fill the tank seperately. Find the time in which each tap can seperately fill the  tank.
\item A pole has to be erected at a point on the boundary of a circular park of diameter $1.3$ metres in such a way that the difference of its distances from two diametrically opposite fixed gates A and B on the boundary is $7$ metres. Is it possible to do so? If yes, at what distances from the two gatees should the pole be erected?
\item Sum of the areas of two squares is $468m^2$. If the difference of their perimeter is 24m, find the sides of the two squares.  
\item Is it possible to design a rectangular mango grove whose length is twice its breadth, and the area is  $800m^2$? If so, find its length and breadth.
\item Is the following situation possible? If so, determine their present ages.
\\ The sum of the ages of the two friends is 20 years. Four years ago, the product of their ages in years was $48$.
\item Is it possible to design a rectangular park of perimeter 80m and area of $400m^2$. If so, find its length and breadth.
\end{enumerate}

\subsection{Formulae}
\input{formulae/ap.tex}
\subsection{CBSE}
\input{cbse/ap.tex}
\subsection{JEE}
\input{jee/ap.tex}
\section{Geometric Progression}
\subsection{Formulae}
\input{formulae/gp.tex}
\subsection{NCERT}
\input{ncert/gp.tex}
\subsection{JEE}
\input{jee/gp.tex}
\section{$Z$ Transform}
\subsection{Formulae}
\input{formulae/zt.tex}
\subsection{NCERT}
\input{ncert/zt.tex}
\subsection{JEE}
\input{jee/zt.tex}
\section{Miscellaneous}
\subsection{NCERT}
\input{ncert/misc.tex}
\subsection{JEE}
\input{jee/misc.tex}
\section{Binomial Theorem}
\subsection{NCERT}
\input{ncert/binom.tex}
\subsection{JEE}
\input{jee/binom.tex}
%\input{jee/exp.tex}
\section{Others}
\subsection{JEE}
%\input{jee/other.tex}
\input{jee/exp.tex}
\subsection{CBSE}
\input{cbse/hd.tex}
\subsection{JEE}
\input{JEE/hd.tex}
%
\section{Triangle}
\subsection{NCERT}
\input{ncert/tri.tex}
\subsection{CBSE}
\input{cbse/tri.tex}
\subsection{JEE}
 \input{JEE/tri.tex}
\subsection{Olympiad}
\input{olympiad/tri.tex}
 %
\section{Circle}
\subsection{NCERT}
\input{ncert/circle.tex}
\subsection{JEE}
\input{JEE/circle.tex}
\subsection{Olympiad}
\input{olympiad/circle.tex}
%
\section{Identities}
\subsection{NCERT}
 Verify
\begin{enumerate}[label=\thesubsection.\arabic*.,ref=\thesubsection.\theenumi,resume*]
	\item $x^3+y^3 = \brak{x+y}\brak{x^2-xy+y^2}$
	\item $x^3-y^3 = \brak{x-y}\brak{x^2+xy+y^2}$
	\item $x^3+y^3+z^3-3xyz = \frac{1}{2}\brak{x+y+z}\sbrak{\brak{x-y}^{2}+\brak{y-z}^{2}+\brak{z-x}^{2}}$
\end{enumerate}
Factorize each of the following
\begin{enumerate}[label=\thesubsection.\arabic*.,ref=\thesubsection.\theenumi]
	\item $8a^3+b^3+12a^2b+6ab^2$
	\item $8a^3-b^3-12a^2b+6ab^2$
	\item $27-125a^3-135a+225a^2$
	\item $64a^3-27b^3-144a^2b+108ab^2$
	\item $27p^3-\frac{1}{216}-\frac{9}{2}p^2 + \frac{p}{4}$
	\item $27y^3+125z^3$ 
	\item $64m^3-343n^3$
	\item $27x^3+y^3+z^3-9xyz$
\end{enumerate}
Find the value of each of the following 
\begin{enumerate}[label=\thesubsection.\arabic*.,ref=\thesubsection.\theenumi,resume*]
	\item $\brak{-12}^3+\brak{7}^3+\brak{5}^3$
	\item $\brak{28}^3+\brak{-15}^3+\brak{-13}^3$
\end{enumerate}
Give possible expressions for the length and breadth of each of the following rectangles, in which their areas are given
\begin{enumerate}[label=\thesubsection.\arabic*.,ref=\thesubsection.\theenumi,resume*]
	\item $25a^2-35a+12$
	\item $35a^2+13y-12$
\end{enumerate}
What are the possible expressions for the dimensions of the cuboids whose volumes are given below
\begin{enumerate}[label=\thesubsection.\arabic*.,ref=\thesubsection.\theenumi,resume*]
	\item $3x^2-12x$
	\item $12ky^2+8ky-20k$
\end{enumerate}

\subsection{CBSE}
 Verify
\begin{enumerate}[label=\thesubsection.\arabic*.,ref=\thesubsection.\theenumi,resume*]
	\item $x^3+y^3 = \brak{x+y}\brak{x^2-xy+y^2}$
	\item $x^3-y^3 = \brak{x-y}\brak{x^2+xy+y^2}$
	\item $x^3+y^3+z^3-3xyz = \frac{1}{2}\brak{x+y+z}\sbrak{\brak{x-y}^{2}+\brak{y-z}^{2}+\brak{z-x}^{2}}$
\end{enumerate}
Factorize each of the following
\begin{enumerate}[label=\thesubsection.\arabic*.,ref=\thesubsection.\theenumi]
	\item $8a^3+b^3+12a^2b+6ab^2$
	\item $8a^3-b^3-12a^2b+6ab^2$
	\item $27-125a^3-135a+225a^2$
	\item $64a^3-27b^3-144a^2b+108ab^2$
	\item $27p^3-\frac{1}{216}-\frac{9}{2}p^2 + \frac{p}{4}$
	\item $27y^3+125z^3$ 
	\item $64m^3-343n^3$
	\item $27x^3+y^3+z^3-9xyz$
\end{enumerate}
Find the value of each of the following 
\begin{enumerate}[label=\thesubsection.\arabic*.,ref=\thesubsection.\theenumi,resume*]
	\item $\brak{-12}^3+\brak{7}^3+\brak{5}^3$
	\item $\brak{28}^3+\brak{-15}^3+\brak{-13}^3$
\end{enumerate}
Give possible expressions for the length and breadth of each of the following rectangles, in which their areas are given
\begin{enumerate}[label=\thesubsection.\arabic*.,ref=\thesubsection.\theenumi,resume*]
	\item $25a^2-35a+12$
	\item $35a^2+13y-12$
\end{enumerate}
What are the possible expressions for the dimensions of the cuboids whose volumes are given below
\begin{enumerate}[label=\thesubsection.\arabic*.,ref=\thesubsection.\theenumi,resume*]
	\item $3x^2-12x$
	\item $12ky^2+8ky-20k$
\end{enumerate}

\subsection{JEE}
 Verify
\begin{enumerate}[label=\thesubsection.\arabic*.,ref=\thesubsection.\theenumi,resume*]
	\item $x^3+y^3 = \brak{x+y}\brak{x^2-xy+y^2}$
	\item $x^3-y^3 = \brak{x-y}\brak{x^2+xy+y^2}$
	\item $x^3+y^3+z^3-3xyz = \frac{1}{2}\brak{x+y+z}\sbrak{\brak{x-y}^{2}+\brak{y-z}^{2}+\brak{z-x}^{2}}$
\end{enumerate}
Factorize each of the following
\begin{enumerate}[label=\thesubsection.\arabic*.,ref=\thesubsection.\theenumi]
	\item $8a^3+b^3+12a^2b+6ab^2$
	\item $8a^3-b^3-12a^2b+6ab^2$
	\item $27-125a^3-135a+225a^2$
	\item $64a^3-27b^3-144a^2b+108ab^2$
	\item $27p^3-\frac{1}{216}-\frac{9}{2}p^2 + \frac{p}{4}$
	\item $27y^3+125z^3$ 
	\item $64m^3-343n^3$
	\item $27x^3+y^3+z^3-9xyz$
\end{enumerate}
Find the value of each of the following 
\begin{enumerate}[label=\thesubsection.\arabic*.,ref=\thesubsection.\theenumi,resume*]
	\item $\brak{-12}^3+\brak{7}^3+\brak{5}^3$
	\item $\brak{28}^3+\brak{-15}^3+\brak{-13}^3$
\end{enumerate}
Give possible expressions for the length and breadth of each of the following rectangles, in which their areas are given
\begin{enumerate}[label=\thesubsection.\arabic*.,ref=\thesubsection.\theenumi,resume*]
	\item $25a^2-35a+12$
	\item $35a^2+13y-12$
\end{enumerate}
What are the possible expressions for the dimensions of the cuboids whose volumes are given below
\begin{enumerate}[label=\thesubsection.\arabic*.,ref=\thesubsection.\theenumi,resume*]
	\item $3x^2-12x$
	\item $12ky^2+8ky-20k$
\end{enumerate}

\section{Equations}
\subsection{NCERT}
 \input{ncert/eq.tex}
\subsection{CBSE}
 \input{cbse/eq.tex}
\subsection{JEE}
 \input{JEE/eq.tex}
 %\input{JEE/zchapters/trigonometric-functions-equations.tex}
\section{Inequalities}
\subsection{NCERT}
\input{./chapters/trig/ineq}
\subsection{JEE}
\input{JEE/ineq}
%\appendices
\fi
\end{document}

